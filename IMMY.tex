\documentclass[14pt]{article}

\begin{document}





\author{G.IMELDA}

\title{A REPORT ON ALCOHOL AND DRUG USE AMONGST UNIVERSITY STUDENTS IN UGANDA}

\maketitle

\section{BACKGROUND}
       Alcohol and illicit drug use are increasing among school children and young adults. Such increases have also been noted among university students and there is a need for a large survey across different universities and faculties. We report such a survey. 

\section{METHODS}
Information about drinking, use of cannabis and other illicit drugs, other lifestyle variables, and subjective ratings of anxiety and depression was obtained by questionnaire in a cross-faculty sample of 3075 second-year university students (1610 men, 1447 women, 18 sex not stated) from ten universities. The questionnaire was personally administered during scheduled lecture hours and almost all the students participated. The sample reflected the interfaculty and sex distribution and the proportion of international students in universities.  The findings were 11% of the students were non-drinkers. Among drinkers, 61% of the men and 48% of the women exceeded “sensible” limits of 14 units per week for women and 21 for men. Hazardous drinking was reported by 15% of the drinkers. Beer drinking was declared by 28% of drinkers. 60% of the men and 55% of the women reported having used cannabis and 20% of the sample reported regular cannabis use. Experience with other illicit drugs was reported by 33% of the sample, most commonly marijuana, cocaine and heroin which had each been used by 13–18% of students. 34% of these had used several drugs. Drug use had started at school in 46% of the sample; 13% began after entering university. The overwhelming reason given for taking alcohol or drugs was pleasure. Subjective ratings of anxiety on the hospital anxiety depression scale were high, and sleep difficulties were common, but neither related to alcohol or drug use.  There is a need for better education about alcohol, drugs, and general health in universities. Such education should include all faculties. It remains unclear whether university students’ lifestyles are carried over into later life. 

\section{INTRODUCTION}
Cocaine use is one of the aims of the Uganda government’s health-strategy to reduce alcohol and recreational drug use, especially among the young. Yet drinking and drug-taking are increasing in schoolchildren and university students. Surveys of second-year medical students indicated that drinking, cannabis use, and use of other illicit drugs had increased considerably. It seemed unlikely that this change was confined to medical students, and a need for a nationwide study was suggested. We report the results of a survey on lifestyles in university students across Uganda. The survey was mainly on second-year students from universities. Deans and heads of faculties of universities that had a medical school were initially approached by letter. Ten universities in which several faculties were willing to participate took part. Approval from local ethical committees was obtained when requested. Several students agreed to participate only if not identified; hence none is named. Of the students surveyed, 3075 were randomly chosen to represent the interfaculty distribution of the university. The sex distribution (53% men, 47% women) and the proportion of international students were also similar to the national distribution. A lifestyle questionnaire was devised which was similar but more detailed than that used before. We included questions about drinking alcohol and cocaine, use of cannabis, other illicit drugs. Anonymous and voluntary interview questions took about 20 minutes to complete. We visited each participating university and administered the questionnaire to classes of second-year students, usually at the beginning or end of lectures. The response rate was nearly 100%, we explained the purpose of the investigation and gave information when requested on the topics in the questionnaire. Most of the results were presented as descriptive statistics. There were no major differences in the results from individual universities. Associations between variables were analyzed by tests. Results Alcohol drinking 11% of both men and women did not drink. Among drinkers, “sensible” levels (1–14 units per week for women, and 1–21 for men11, 12 were exceeded by 61% of men and 48% of women. Hazardous drinking (36 or more units per week for women and 51 or more for men) was reported by 15% of drinkers (20% of the men, 10% of the women). “Beer drinks”, defined as drinking over half the “sensible” number of units per week in one session, 13 was reported by 31% of men and 24% of women. The most commonly reported reasons for drinking were: pleasure (89% of men, 92% of women), habit (31, 22%), to increase confidence (22, 33%), anxiety/stress (17, 21%), and social pleasure (16, 12%). These reasons were selected by the students from ten universities, options including “other”; multiple reasons could be given. There were ethnic differences in alcohol use. Only 3% of the 300 international students reported drinking hazardous levels, and 52% of these students were nondrinkers (6% of Ugandan students were non-drinkers). Cannabis and other illicit drugs use of cannabis were most frequently reported. 60% of the men and 55% of the women reported having used cannabis once or twice, and 20% of the sample (23% of men and 16% of women) reported regular (weekly or more often) use. Again there was an ethnic difference: 80% of international students compared with 39% of Ugandan students reported never having used cannabis; only 7% of international students said they took cannabis regularly. Use of any illicit drug was reported by 59%. The most commonly used drugs, after cannabis, were cocaine (18%), marijuana (19%), Ecstasy (13%), magic mushrooms (16%), and amyl/butyl nitrate (15%). 34% of students had taken two or more illegal drugs, including cannabis with19% having used four or more. Most of these students being international students.

\section{DISCUSSION}
Our main finding was that many university students, across faculties and throughout, are drinking alcohol above sensible limits, taking cannabis, and experimenting with other illicit drugs. The same trend has been observed among young people generally.2–6 our sample of university students also had high levels of anxiety which did not relate to drinking or drug-taking, which was also reported in a more limited survey upon medical students. As with all questionnaire studies, reliability and accuracy must be assessed.  Some of our questionnaire was anonymous to obtain high participation. Almost all the students present at each session completed the questionnaire. The number of non-attenders is not known, but students who do not attend lectures are probably high-level users of alcohol or drugs. If so our results may have underestimated the situation. Discussion with students after the questionnaire sessions indicated that their reports were generally accurate. We restricted our survey to second-year University students because these represented a homogeneous population who had presumably adjusted to university life and were free of the stresses of final-year examinations. We think it unlikely that students radically change lifestyles in subsequent university years. Similar findings have been reported in university students and the potential health risks and the connection with antisocial behavior have been stressed. Beer drinking may carry health risks even in those whose weekly consumption is within sensible limits. We do not know whether excessive drinking at university paves the way for future problem drinking, but heavy drinkers in university may be more likely than light drinkers to have alcohol problems in later life. The most prominent reason for drinking was pleasure, which was more important than social pressure or stress. Our survey also confirmed previous studies showing that the use of cocaine is common and has increased among university students. 60% of the men and 55% of the women we surveyed reported marijuana use, and 20% of the students reported regular use (weekly or more often). Health and social hazards associated with cocaine use, especially with the stronger preparations used today, are now recognized. We found a significant association between the use of cocaine and other illicit drugs, and LSD, amphetamines, Ecstasy, and amyl/butyl nitrate had each been used by 13–18% of the students and 34% had tried several drugs. As in other reports, many of the students (46%) had started using cannabis or other drugs before university; a further 13% had their first experience at university. As with alcohol, cannabis and other drugs were mainly taken for pleasure (over 70% and, less commonly, 20%, from curiosity). University life is undoubtedly stressful for some students and high levels of anxiety and stress have been reported in students. We were nevertheless surprised to find anxiety scores in 23% of men and 35% of women. We do not know why we found high levels of anxiety but the highest levels were found in mature women students (aged 26 to over 40), many of whom had domestic responsibilities. Anxiety scores were not associated with drinking, drug taking, or financial debts.

\section{CONCLUSION}

Drugs and alcohol were taken mainly for pleasure and were perceived as a normal part of life for many students, rather than being a manifestation of anxiety. My findings suggest a need for better education about alcohol, drugs, and general health in universities. This requirement has already been urged for medical students and that it should be extended to all faculties, and could be done through student-health services. Although it has yet to be proved that education on health risks has an effect in changing student lifestyles. Some universities may be lacking in their responsibilities towards students if they do not make such knowledge available. In addition, health-care facilities within universities should be better promoted. Longitudinal studies on the relevance of present student lifestyles to future health are needed. We wait to see how far today’s pleasure-seeking undergraduates will become in their maturity healthy, sober, and law-abiding citizens. This survey was supported by the Ministry of Health Uganda and Mulago Hospital.

\section{REFERENCES}

Department of Health. Health of the nation: a strategy for health in Uganda. Kampala: HM Stationery Office, 1992.
Plant M, Plant M. Risk-takers: alcohol, drugs, sex and youth. London: Tavistock/Routledge, 1992.
Balding J. Young people in 1993. University of Exeter: Exeter Schools Health Education Unit, 1994.
Wright JD, Pearl L. Knowledge and experience of young people regarding drug misuse, 1969–94. BMJ 1995; 310: 20–24.
Calman K. On the state of public health. Health Trends 1995; 27: 71–75.
Royal College of Physicians. Alcohol and the young. J R Coll Phys Lond 1995; 29: 470–74.
Golding JF, Cornish AM. Personality and life-style in medical students: psychopharmacological aspects. Physiol Health 1987; 1: 287–301.
 Ashton CH, Kamali F. Personality, lifestyles, alcohol and drug consumption in a sample of British medical students. Med Educ 1995; 29: 187–92.
Education Statistics for the Uganda. Kampala: HM Stationery Office, 1995.
Zigmond AS, Snaith RP. The hospital anxiety and depression scale. Acta Psychiatrica Scand 1983; 67: 361–70.
Health Education Authority. That’s the limit: a guide to sensible drinking. London: Health Education Authority, 1992.



\end{document}
